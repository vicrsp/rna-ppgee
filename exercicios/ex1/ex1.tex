\documentclass[conference]{IEEEtran}
% If the IEEEtran.cls has not been installed into the LaTeX system files, 
% manually specify the path to it:
% \documentclass[conference]{../sty/IEEEtran} 
\usepackage[brazil]{babel}
\usepackage{amsmath}
\usepackage{multirow}
\usepackage[utf8]{inputenc}
\usepackage[T1]{fontenc}

% correct bad hyphenation here
\hyphenation{op-tical net-works semi-conduc-tor IEEEtran}


\begin{document}
	
	% paper title
	\title{Redes Neurais Artificiais - Exercício 1}
	
	
	% author names and affiliations
	% use a multiple column layout for up to three different
	% affiliations
	\author{\authorblockN{Victor São Paulo Ruela \\}
		\authorblockA{Programa de Pós-Graduação em Engenharia Elétrica\\
			Universidade Federal de Minas Gerais\\
			Belo Horizonte, Brasil\\
            Email: victorspruela@gmail.com}}
	
	% avoiding spaces at the end of the author lines is not a problem with
	% conference papers because we don't use \thanks or \IEEEmembership
	
	% use only for invited papers
	%\specialpapernotice{(Invited Paper)}
	
	% make the title area
	\maketitle
	
	\begin{abstract}
			
		Neste exercício é feita um resumo e análise crítica do artigo ``Improving generalization of MLPs with multi-objective optimization''.
		
	\end{abstract}


    \begin{thebibliography}{99}
        \bibitem{mobj} de Albuquerque Teixeira, R., Braga, A. P., Takahashi, R. H., \& Saldanha, R. R. (2000). Improving generalization of MLPs with multi-objective optimization. Neurocomputing, 35(1-4), 189-194.
    \end{thebibliography}
    
	
\end{document}